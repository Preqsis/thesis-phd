V této publikaci se soustředíme na modelování \emph{Kataklyzmických proměnných} hvězd (CV - z angl. \emph{Cataclysmic Variables}). Konkrétně na dynamiku akrece v těchto binárních systémech. CV hvězdy jsou známé tím, že jejich světelné křivky mohou vykazovat vysoce aperiodické chování. Tento fenomén, nazývaný \emph{flickering}, poprvé objevili \citep{henize1949} a \citep{lenouvel1954}. Původ a vnitřní mechanizmy, které dávají vzniknout těmto nepravidelným změnám jasnosti, nejsou stále dobře pochopeny. První teorie připisovaly vznik flickeringu vysoce turbuletní povaze místa, kde se proud plynu ze sekundární hvězdy setkává s akrečním diskem, také nazývanému \emph{horká skvrna} (angl. \emph{hot spot}). Některé pozorované systémy vykazují nejvyšší urovně flickeringu v době, kdy horká skvrna není zakryta před pozorovatelem. Pozdější studie, například \citep{patterson1981} nebo \citep{wood1986}, ale ukázaly, že pro značnnou část CV systémů neexistuje  korelace mezi orbitální fází a urovněmi flickeringu. Během let bylo vytvořeno mnoho modelů, například \citep{dobrotka2012}, \citep{kley1997}, \citep{lyubarskii1997} nebo \citep{yonehara1997}. Žádný ale zatím neposkytl definitivní vysvětlení původu flickeringu.

Jako první cíl našeho výzkumu jsme implementovali vlastní numerický model akrečního disku, nazývaný \emph{Multilayer Dripping Handrail} (MDH - česky. \emph{Vícevrstvé Kapající Zábradlí}), který jsme použili ke studiu dynamiky distribuce hmoty a extrakci světelných křivek závislých na nastavených parametrech a počátečních podmínkách. Poskytli jsme vysoce přízpůsobitelný a dobře optimalizovaný nástroj k modelování akrečního disku, který navíc nevyžaduje superpočítač.

Za druhý cíl jsme si stanovili získat hodnotu volného parametru $\alpha$ ze standardního Shakura-Sunyaevova $\alpha$-disk modelu pro námi modelovaný CV systém. Tuto analýzu jsme založili na výsledcích numerických simulací MDH modelu, které se ukázaly být v dobré shodě s analytickým řešením $\alpha$-disk modelu. Proto jsme schopni přesně určit parametr $\alpha$ a spárovat jej s CV systémem specifických parametrů.

Text této publikace začínáme stručným přehledem různých typů ve vesmíru pozorovatelných akrečních systémů a navázané teorie. Poté se zaměřujeme na teorii hlavního mechanizmu MDH, který využíváme jako spouštěč redistribuce hmoty v disku. Tento kritický mechanizmus je založen na modelech kapajícího kohoutku, konkrétně \emph{pružinového} (MSM - angl. \emph{Mass-Spring Model}), který zavedl \citep{shaw1984}.

Dále předkládáme definice a podrobnější vysvětlení podkladové teorie našeho MDH modelu, a také si procházíme konkrétní implementaci modelového kódu v jazyce \CC, její koncepce a použité knihovny. 

V závěru demonstrujeme možnosti a výsledky simulací několika různých konfigurací modelu pro stejný generický CV systém a diskutujeme charakteristiky různých výsledků. Výsledná data ze zmíněných simulací jsme poté využili k získání požadovaného $\alpha$ parametru.
