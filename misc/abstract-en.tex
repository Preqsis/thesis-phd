In this publication, we focused on the numerical modeling of Cataclysmic Variables (CV) and, in particular, the accretion system dynamics of these binary systems. CVs are known to exhibit aperiodic highly aperiodic behavior in the light curve. This phenomenon, called \emph{flickering}, was first discovered by \citep{henize1949} and \citep{lenouvel1954}. To this day, its origins or the underlying mechanism, which produces the brightness irregularities, are not well understood. First, it was theorized that the flickering could be attributed to the highly turbulent nature of the accretion disc region, where the gas stream from the secondary star, also called the \emph{hot spot}, meet the accretion disc's body. Some observed systems exhibit the flickering the most when the hot spot is not obscured from the observer. However, later studies, like \citep{patterson1981} or \citep{wood1986}, have demonstrated that for other significant portion of CV systems, there is no correlation between the orbital phase and the flickering levels. There were many models created over the years, for example, \citep{dobrotka2012}, \citep{kley1997}, \citep{lyubarskii1997}, \citep{yonehara1997}. However, none could provide definitive explanations of the flickering's origin.  

As the first goal of our research, we implemented a custom numerical accretion disc model, called \emph{Multilayer Dripping Handrail} (MDH), which we use to study the dynamics of matter distribution and extract light curves of different characteristics depending or set simulation parameters and initial conditions. We provided a highly customizable and well-optimized tool for accretion disc modeling, which does not require a supercomputer. 

The second goal we set up was to obtain the value of free parameter $\alpha$ from the standard Shakura-Sunyaev $\alpha$-disc model (see \citep{shakura1973}) for modeled CV systems. We based the analysis on the numerical simulation results of MDH, which are in good agreement with the analytical $\alpha$-dics solution. Therefore, we can accurately obtain the $\alpha$ parameter and pair it to the specific CV system parameters. 

We start this publication's text with a brief overview of different types of accreting systems observable in the universe and the underlying theory. Then we go over the theory behind the primary mechanism of MDH, which we use to trigger the matter redistribution in the disc. The critical \emph{dripping} mechanism is based on the dripping faucet models and, specifically, the \emph{Mass-Spring Model} first introduced by \citep{shaw1984}. 

Then, we provide a definition and explanation of the theory behind our MDH model, and also we go over the actual \CC\ code implementation, its concepts, limitations, and used libraries. 

In the end, we demonstrate our model's capabilities and simulation results on several different configurations for the same generic CV system and discuss the characteristics of different outcomes. Also, we use the data obtained from the aforementioned simulations to get the desired $\alpha$ parameter value. 
