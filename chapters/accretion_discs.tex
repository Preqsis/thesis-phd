\chapter{ACCRETION DISCS}

The term \emph{accretion} comes from a Latin word \emph{accrescere} which literarily means \emph{become larger}, and in astronomy it is used to refer exactly to that process. That is the \emph{coming together and cohesion of matter under the influence of gravity to form larger bodies}. One could very easily argue, that that is one of the most basic processes in universe. From the largest galaxies to the smallest rocks floating around in the solar system, all the stars and planets, basically all there is and all we see around us every day, at some point in the past was smashed together by gravity. Atom by atom, piece by piece to form larger and larger structures. Even the dinosaurs most probably met their fate by a city sized asteroid that accreted Earth some 65 million years ago. 

Accretion is not only the mass moving and colliding, but it is also energy taking different forms in the process, often many different kinds of radiation.

\section[$\alpha$-Disc model]{$\alpha$-Disc model}

\begin{align}
\begin{split}
\Sigma 	&= 5.2 \alpha^{-4/5} \dot{M}^{7/10}_{16} m^{1/4}_1 R^{-3/4}_{10} f^{14/5}\ \mathrm{g\ cm^{-2}}, \\
H		&= 1.7 \times 10^8 \alpha^{-1/10} \dot{M}^{3/20}_{16} m^{-3/8}_1 R^{9/8}_{10} f^{3/5}\ \mathrm{cm}, \\
\rho		&= 3.1 \times 10^{-8} \alpha^{-7/10} \dot{M}^{11/20}_{16} m^{5/8}_1 R^{-15/8}_{10} f^{11/5}\ \mathrm{g\ cm^{-3}}, \\
T_c		&= 1.4 \times 10^4 \alpha^{-1/5} \dot{M}^{3/10}_{16} m^{1/4}_1 R^{-3/4}_{10} f^{6/5}\ \mathrm{K}, \\
\tau		&= 190 \alpha^{-4/5} \dot{M}^{1/5}_{16} f^{4/5}, \\
\nu		&= 1.8 \times 10^{14} \alpha^{4/5} \dot{M}^{3/10}_{16} m^{3/4}_1 R^{3/4}_{10} f^{6/5}\ \mathrm{cm^2\ s^{-1}},  \\
v_R		&= 2.7 \times 10^{14} \alpha^{4/5} \dot{M}^{3/10}_{16} m^{-1/4}_1 R^{-1/4}_{10} f^{-14/15}\ \mathrm{cm\ s^{-1}},  \\
\mathrm{with}\ f		&= \left[ 1 - \left( \frac{R_*}{R} \right)^{1/2} \right]^{1/4}. \\
\end{split}
\label{eq:alpha_model}
\end{align}
