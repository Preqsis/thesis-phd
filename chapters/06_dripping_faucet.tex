\chapter{DRIPPING FAUCET}
\label{chap:dripping_faucet}



\section{Fluid dynamical model (FDM)}
\section{Mass-spring model (MSM)} \label{section:msm}

\thispagestyle{empty}


\begin{align}
	\D{}{t} \left(m \D{z}{t}\right) &= -kz - \gamma\D{z}{t} + m f_{\text{g}}, \label{eq:ode_1} \\
	\D{m}{t} &= Q, \label{eq:ode_2}
\end{align}

where $z$ represents the position of the hanging mass $m$. $Q$ is \emph{mass influx}, and it is the main determining parameter of MSM because by choosing a specific value of $Q$, we can drastically alter our model's behavior, as shown in \cite{msmm1999}. Parameter $\gamma=0.05$ is the dampening ratio, and $k$ is the stiffness of the imaginary spring defined as

\begin{equation}
    \begin{aligned}
        & k~= 
        \begin{cases}
            -11.4\ m + 52.5 \hspace{10mm} (m < 4.61) \\
            \hspace{12mm} 0 \hspace{20mm} (m \ge 4.61 ).
        \end{cases}
    \end{aligned}
    \label{eq:spring_stiffness}
\end{equation}

Relations expressed by \eqref{eq:spring_stiffness} and the value of $\gamma$ were obtained experimentally by \cite{shaw1984} and provide a good description for the real-world behavior of drops and leaky faucet. Simply put, \eqref{eq:spring_stiffness} means that if the mass contained in the MSM exceeds the value of $m = 4.61$, the matter essentially goes into free fall. 
