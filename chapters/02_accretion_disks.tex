\chapter{ACCRETION DISKS}

The term \emph{accretion} comes from a Latin word \emph{accrescere} which literarily means \emph{become larger}, and in astronomy, we refer exactly to that process. That is the \emph{coming together and cohesion of matter under the influence of gravity to form larger bodies}. One could easily argue that it is one of the most fundamental processes in the universe. From the giant galaxies to the tiniest rocks floating around in the solar system. All the stars, planets, and all there is were smashed together by gravity at some point in the past. Atom by atom, piece by piece, to form larger and larger structures. Even the dinosaurs probably met their fate by a city-sized asteroid that accreted Earth some 65 million years ago. 

Accretion is not only the mass moving and colliding but also energy taking different forms in the process. Depending on the nature of the accreting system and its central object, many types of radiation escape the swirling vortex of gas and particles we call \emph{accretion disk}. If we sort accreting astrophysical systems based on their size and radiation power, we can classify them as follows.  

\section{Active galactic nuclei (AGN) and Quasar accretion}

As \emph{active galactic nucleus} (AGN), we refer to a central high-luminosity region containing a supermassive black hole in most galaxies' center. The radiative power of the AGN is usually higher than that of the whole host galaxy, and the radiation characteristics indicate that stars are not the main source of this radiation. Instead, mass accretion onto the central supermassive black hole is the more likely source of this excess non-stellar radiation. 

The main distinguishing characteristic of AGNs is whether they are \emph{radio loud} or \emph{radio quiet}, which depends on the existence of jets. By \emph{jets}, we mean relatively narrow streams of accreted mass ejected from the black hole in both directions, roughly colinear with its axis of rotation. These mass ejections can travel at relativistic speeds and reach thousands of light years away from the source object. 

\section{X-ray binaries (XB) accretion}

\section{Young stellar objects (YSO) accretion}

\section{Cataclysmic variables (CV) accretion}

\section[Shakura-Sunyaev $\alpha$-Disc model]{Shakura-Sunyaev $\alpha$-Disc model}
\begin{align}
\begin{split}
\Sigma 	&= 5.2 \alpha^{-4/5} \dot{M}^{7/10}_{16} m^{1/4}_1 R^{-3/4}_{10} f^{14/5}\ \mathrm{g\ cm^{-2}}, \\
H		&= 1.7 \times 10^8 \alpha^{-1/10} \dot{M}^{3/20}_{16} m^{-3/8}_1 R^{9/8}_{10} f^{3/5}\ \mathrm{cm}, \\
\rho		&= 3.1 \times 10^{-8} \alpha^{-7/10} \dot{M}^{11/20}_{16} m^{5/8}_1 R^{-15/8}_{10} f^{11/5}\ \mathrm{g\ cm^{-3}}, \\
T_c		&= 1.4 \times 10^4 \alpha^{-1/5} \dot{M}^{3/10}_{16} m^{1/4}_1 R^{-3/4}_{10} f^{6/5}\ \mathrm{K}, \\
\tau		&= 190 \alpha^{-4/5} \dot{M}^{1/5}_{16} f^{4/5}, \\
\nu		&= 1.8 \times 10^{14} \alpha^{4/5} \dot{M}^{3/10}_{16} m^{3/4}_1 R^{3/4}_{10} f^{6/5}\ \mathrm{cm^2\ s^{-1}},  \\
v_R		&= 2.7 \times 10^{14} \alpha^{4/5} \dot{M}^{3/10}_{16} m^{-1/4}_1 R^{-1/4}_{10} f^{-14/15}\ \mathrm{cm\ s^{-1}},  \\
\mathrm{with}\ f		&= \left[ 1 - \left( \frac{R_*}{R} \right)^{1/2} \right]^{1/4}. \\
\end{split}
\label{eq:alpha_model}
\end{align}
