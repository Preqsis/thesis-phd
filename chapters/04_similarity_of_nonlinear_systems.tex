\chapter{MODELING COMPLEX SYSTEMS}
\label{chap:similary_of_nonlinear_systems}
\thispagestyle{empty}

\mquote{...}{...}

Regardless of the area of scientific study, wheter it is physics, biology, sociology or anyhing else, we often observe a highly complex system just by the means of some macrosopic dataset without understanding or any knowlege about the microscopic (i.e., the inner workings) mechanism. We could say, that this is the actuall reason for the existence of any science and scientific method, and most of what any reasearcher in any area do is to create models of the real world. We could break the process of the model construction in to three basic steps. 

\begin{enumerate}[label=\alph*)]
    \item{Observation of macrosopic system behavior.}
    \item{Microscopic model hypothesis.}
    \item{Validation of modeled data.}
\end{enumerate}

The process is quite straightforward. You accuire some data, wich you do not understand or know the origin of, you post a hypothesis (i.e., create a model) and then you validade the results of your model against the initial observation. And you repeat the steps b) and c) until your model's results satisfy the observational data. 

% \begin{figure}[H]
% \begin{center}
% \begin{tikzpicture}[node distance=2cm]
    % \node(pr1)[process]{1) Observation of macrosopic system behavior.};
    % \node(pr2)[process, below of=pr1]{2) Microscopic model hypothesis.};
    % \node(pr3)[process, below of=pr2]{3) Validation of modeled data.};
%
    % \draw[arrow] (pr1) -- (pr2);
    % \draw[arrow] (pr2) -- (pr3);
    % \draw[arrow] (pr3) -| (pr2);
% \end{tikzpicture}
% \end{center}
% \end{figure}


% magnetorotacni nestabilita se uvazuje jako pruzina
% msmm je taky pruzina


