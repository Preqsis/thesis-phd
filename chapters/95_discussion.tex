\chapter{CONCLUSION}
\thispagestyle{empty}
\mquote{The greatest teacher, failure is.}{Master Yoda}

    Throughout this publication, we provided a detailed description of our \emph{Multilayer Accretion Disc} model (MDH) (see Chapter~\ref{chap:multilayer_dripping_handrail}), did an overview of its specific code implementation (see Chapter~\ref{chap:model_implementation}), and demonstrated its capabilities on chosen generic CV system using several model configurations (see Chapter~\ref{chap:cv_models}). We then used these results to obtain specific values of the Shakura-Sunyaev $\alpha$ parameter for each configuration (see Chapter~\ref{chap:alpha_model_fitting}). 

    In this last Chapter, we would like to discuss the successes and limitations regarding the specific topics of our research journey with the MDH model. Also, we want to provide some questions, ideas, or inspirations for our future research or the related research of others.
    
\section{Simulations results}
    In Chapter~\ref{chap:cv_models}, we presented a few result examples of MDH with different free parameters $q$ (local MSM mass inflow) and $\psi$ (the drop breakout ratio). We intentionally chose the extreme combinations of high a low values to demonstrate the dependence of MDH behavior on free parameter values. 

    Figures~\ref{fig:plot_density_temperature_c1}~through~\ref{fig:plot_density_temperature_c7} demonstrate perfectly the possible result alterations by the use of different free parameters. However, one crucial feature is more or less prevalent in all results, and that is the radial mass distribution in the accretion disk's body. 

    In the extracted area density $\Sigma$ and the mean area density $\bar{\Sigma}$, we can see a gradual increase towards the center and a sudden drop in the amount of mass contained within the cells. This drop in mass density happens relatively close to the inner edge of the accretion disk but not at the very edge of it, which is in good agreement with the Shakura-Sunyaev $\alpha$-disk solutions. Also, the numeric levels of area density do agree with the analytical solutions. This could be considered a good indication that our simulation results are not far-off. Alternatively, let us say a mutual confirmation of ours and Shakura-Sunyaev's models because we arrived at a very similar result by a completely different way of pure numerical simulation.

    We also extracted synthetic light curves for all presented SIM runs (see Figures~\ref{fig:plot_light_curves_undisturbed}~and~\ref{fig:plot_light_curves_disturbed}). Each of these light curves exhibits vastly different characteristics, from the periodic brightening visible in C4 to the aperiodic C3 and C2 light curves.  

    Unfortunately, we are limited by this publication's written format, forcing us to present only still visualizations of specific simulation frames. Therefore, we also created video visualizations for all SIM runs presented in this study. These should provide a better insight into the MDH's dynamic nature. The videos are available at

    \begin{center}
        \url{https://monoceros.physics.muni.cz/\~kveton/} 
    \end{center}

    And again, please feel free to examine, comment, and share any insight or suggestions on the video visualizations of our results. 

\section{Determining the Shakura-Sunyaev $\alpha$ parameter}
    One of our research's main goals was to devise a method for determining the value of the free $\alpha$ parameter in the standard Shakura-Sunyaev accretion disk model. Our MDH model gave us the tools to do exactly that because we can run a highly customizable accretion disk simulation. By iterative process of varying MDH's free parameters for specific CV systems and analyzing the results, it is possible to get to a tailored model configuration for that specific accretion system. Therefore, the results of such simulation will give us the data from which we can obtain the $\alpha$ parameter for the analyzed real-world system (see Chapter~\ref{chap:alpha_model_fitting}).

    We analyzed three different extreme cases of free parameter variations. A summary of the $\alpha$ parameter results, with the used free parameters and other initial conditions, is listed in Table~\ref{tab:table_alpha_summary} (see also Figure~\ref{fig:plot_mean_density_fit}).
    
    \begin{table}[ht]
    \centering
    \begin{tabular*}{\columnwidth}{@{\extracolsep{\fill}}cccccc}
        SIM run & $q$ & $\psi$ & $\dot{M} [\si{\gram \cdot \second^{-1}}]$ & $\alpha$ & $\alpha\ \mathrm{std.\ dev.}$ \\ 
    \hline\hline
        C2 & $0.1$ & $0.9$ & $10^{14}$ & $0.067$ & $0.003$ \\
        C3 & $0.9$ & $0.1$ & $10^{14}$ & $0.254$ & $0.008$ \\
        C4 & $0.9$ & $0.9$ & $10^{14}$ & $0.994$ & $0.036$ \\
    \hline
    \end{tabular*}
        \caption{Summary of Shakura-Sunyaev's $\alpha$ parameter values for MDH free parameter variations. All results are based on MDH simulations with CV system parameters: $M_{\mathrm{p}} = 0.63 M_{\odot}$, $r_{\mathrm{in}} = 0.01 R_{\odot}$, $r_{\mathrm{out}} = 1.16 R_{}$, $T_{\mathrm{out}} = 4500 \si{\kelvin}$.}
    \label{tab:table_alpha_summary}
    \end{table}

\section{Blob impacts experiments}
    Another feature of our MDH code is the ability to disturb the simulation mid-run by a blob impact onto the accretion disk's body. Conceptually, it is a relatively straightforward operation but a very powerful feature. We can add any size or shape of irregular mass accretion at any point during the simulation, and it opens up a whole new range of simulation cases. Starting with irregular mass ejections from the secondary component, through interfering with the steady accretion stream, to studies of accretion disk reactions on different blob configurations. Moreover, it offers endless possibilities for altering the accretion disk's power output and the light curve. 

    The part of MDH's code responsible for handling the matter blob addition is implemented so the user can define any shape and size configuration and supply it as an easily readable and editable JSON file.

    We did three SIM runs C5, C6, and C7, with the same free parameters settings as C2, C3, and C4. The results of these disturbed SIM runs demonstrate the different effects that the variations of MDH's free parameters have on the simulation outcome (see Figure~\ref{fig:plot_light_curves_disturbed}).

\section{Future research}
    As any other research is never finished, so is ours. In our case, this is slightly exaggerated because a significant portion of work done on MDH is actually a software development project with its own quirks and challenges. We, or anyone interested, can always add, improve or extend the capabilities of our model. Therefore, we want to offer suggestions and inspiration for future research. 

    The first idea that comes to mind is testing more critical dripping mechanisms other than MSM. The MSM turned out to be a good choice because MDH, in its current implementation, produces results that are in very good agreement with Shakura-Sunyaev $\alpha$-disk solutions. It would also be helpful and practical if the dripping mechanism code could be \emph{plugged-in} to the main simulation code for even higher customizability. Currently, the MSM code is part of the main codebase. 

    The MDH underlying theory and definitions, as described in Chapter~\ref{chap:multilayer_dripping_handrail}, could undoubtedly be extended. For example, we could introduce magnetic fields or relativistic effects into the mix. This would make the MDH code more usable for other types of accreting systems.

    Also, the parallelization of MDH uses a relatively basic approach. We could rework certain parts of the code to utilize more specialized hardware, like GPUs. The downside of this is that it would also require the usage of such hardware. Therefore the code would be more limited. We consider the ability to run on common hardware a plus, but the more specialized hardware would certainly extend the simulation capabilities. 
